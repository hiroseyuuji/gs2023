%#!uplatex -kanji=%k
\documentclass[12pt,uplatex,a5]{jsarticle}
\pagestyle{empty}
\usepackage{okumacro}	% required for `\ruby' (yatex added)
\begin{document}
% ここから-------------------
\section*{割引券}
\huge
\begin{tabular}{|c|c|c|}
 \hline
 ::ユーザID:: & \ruby{::氏名::}{::読み仮名::} \\\hline
\end{tabular}
\pagebreak
% ここまでの部分が CSVファイルのレコードに応じてその件数分だけ生成される。
% その結果は out/output.tex に書き込まれる。
\end{document}
